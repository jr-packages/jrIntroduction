\documentclass[]{tufte-handout}

% ams
\usepackage{amssymb,amsmath}

\usepackage{ifxetex,ifluatex}
\usepackage{fixltx2e} % provides \textsubscript
\ifnum 0\ifxetex 1\fi\ifluatex 1\fi=0 % if pdftex
  \usepackage[T1]{fontenc}
  \usepackage[utf8]{inputenc}
\else % if luatex or xelatex
  \makeatletter
  \@ifpackageloaded{fontspec}{}{\usepackage{fontspec}}
  \makeatother
  \defaultfontfeatures{Ligatures=TeX,Scale=MatchLowercase}
  \makeatletter
  \@ifpackageloaded{soul}{
     \renewcommand\allcapsspacing[1]{{\addfontfeature{LetterSpace=15}#1}}
     \renewcommand\smallcapsspacing[1]{{\addfontfeature{LetterSpace=10}#1}}
   }{}
  \makeatother

\fi

% graphix
\usepackage{graphicx}
\setkeys{Gin}{width=\linewidth,totalheight=\textheight,keepaspectratio}

% booktabs
\usepackage{booktabs}

% url
\usepackage{url}

% hyperref
\usepackage{hyperref}

% units.
\usepackage{units}


\setcounter{secnumdepth}{-1}

% citations

% pandoc syntax highlighting
\usepackage{color}
\usepackage{fancyvrb}
\newcommand{\VerbBar}{|}
\newcommand{\VERB}{\Verb[commandchars=\\\{\}]}
\DefineVerbatimEnvironment{Highlighting}{Verbatim}{commandchars=\\\{\}}
% Add ',fontsize=\small' for more characters per line
\newenvironment{Shaded}{}{}
\newcommand{\AlertTok}[1]{\textcolor[rgb]{1.00,0.00,0.00}{\textbf{#1}}}
\newcommand{\AnnotationTok}[1]{\textcolor[rgb]{0.38,0.63,0.69}{\textbf{\textit{#1}}}}
\newcommand{\AttributeTok}[1]{\textcolor[rgb]{0.49,0.56,0.16}{#1}}
\newcommand{\BaseNTok}[1]{\textcolor[rgb]{0.25,0.63,0.44}{#1}}
\newcommand{\BuiltInTok}[1]{#1}
\newcommand{\CharTok}[1]{\textcolor[rgb]{0.25,0.44,0.63}{#1}}
\newcommand{\CommentTok}[1]{\textcolor[rgb]{0.38,0.63,0.69}{\textit{#1}}}
\newcommand{\CommentVarTok}[1]{\textcolor[rgb]{0.38,0.63,0.69}{\textbf{\textit{#1}}}}
\newcommand{\ConstantTok}[1]{\textcolor[rgb]{0.53,0.00,0.00}{#1}}
\newcommand{\ControlFlowTok}[1]{\textcolor[rgb]{0.00,0.44,0.13}{\textbf{#1}}}
\newcommand{\DataTypeTok}[1]{\textcolor[rgb]{0.56,0.13,0.00}{#1}}
\newcommand{\DecValTok}[1]{\textcolor[rgb]{0.25,0.63,0.44}{#1}}
\newcommand{\DocumentationTok}[1]{\textcolor[rgb]{0.73,0.13,0.13}{\textit{#1}}}
\newcommand{\ErrorTok}[1]{\textcolor[rgb]{1.00,0.00,0.00}{\textbf{#1}}}
\newcommand{\ExtensionTok}[1]{#1}
\newcommand{\FloatTok}[1]{\textcolor[rgb]{0.25,0.63,0.44}{#1}}
\newcommand{\FunctionTok}[1]{\textcolor[rgb]{0.02,0.16,0.49}{#1}}
\newcommand{\ImportTok}[1]{#1}
\newcommand{\InformationTok}[1]{\textcolor[rgb]{0.38,0.63,0.69}{\textbf{\textit{#1}}}}
\newcommand{\KeywordTok}[1]{\textcolor[rgb]{0.00,0.44,0.13}{\textbf{#1}}}
\newcommand{\NormalTok}[1]{#1}
\newcommand{\OperatorTok}[1]{\textcolor[rgb]{0.40,0.40,0.40}{#1}}
\newcommand{\OtherTok}[1]{\textcolor[rgb]{0.00,0.44,0.13}{#1}}
\newcommand{\PreprocessorTok}[1]{\textcolor[rgb]{0.74,0.48,0.00}{#1}}
\newcommand{\RegionMarkerTok}[1]{#1}
\newcommand{\SpecialCharTok}[1]{\textcolor[rgb]{0.25,0.44,0.63}{#1}}
\newcommand{\SpecialStringTok}[1]{\textcolor[rgb]{0.73,0.40,0.53}{#1}}
\newcommand{\StringTok}[1]{\textcolor[rgb]{0.25,0.44,0.63}{#1}}
\newcommand{\VariableTok}[1]{\textcolor[rgb]{0.10,0.09,0.49}{#1}}
\newcommand{\VerbatimStringTok}[1]{\textcolor[rgb]{0.25,0.44,0.63}{#1}}
\newcommand{\WarningTok}[1]{\textcolor[rgb]{0.38,0.63,0.69}{\textbf{\textit{#1}}}}

% longtable

% multiplecol
\usepackage{multicol}

% strikeout
\usepackage[normalem]{ulem}

% morefloats
\usepackage{morefloats}


% tightlist macro required by pandoc >= 1.14
\providecommand{\tightlist}{%
  \setlength{\itemsep}{0pt}\setlength{\parskip}{0pt}}

% title / author / date
\title{Practical 1}
\author{Jumping Rivers}
\date{}


\begin{document}

\maketitle




\hypertarget{getting-started}{%
\section{Getting Started}\label{getting-started}}

\hypertarget{start-a-new-script-file}{%
\subsection{Start a new script file}\label{start-a-new-script-file}}

If a new empty R script file does not open automatically when you launch
RStudio you can open a new one\footnote{Alternatively use
  \texttt{Ctrl\ +\ shift\ +\ n} to launch a new R script}

\begin{verbatim}
   File -> New File -> Rscript
\end{verbatim}

\noindent Make sure that you choose the correct type of file. The
RStudio IDE is clever in the way that it treats file extensions.
Choosing a R script file will make sure that when you come to save your
script it is saved as the correct type. It is also necessary to allow us
to easily send R code from the text editor to be evaluated in the R
terminal.

\hypertarget{sending-your-first-code}{%
\subsection{Sending your first code}\label{sending-your-first-code}}

Let's start with sending our first piece of code to be evaluated. In the
script editor (likely the top left window) write a simple piece of code,
say

\begin{Shaded}
\begin{Highlighting}[]
\NormalTok{x =}\StringTok{ }\DecValTok{5}
\end{Highlighting}
\end{Shaded}

\noindent To send this code from the editor to be run by R you have a
few options:

\begin{enumerate}
\def\labelenumi{\arabic{enumi}.}
\tightlist
\item
  If the cursor is on the same line as the code you can either\footnote{Both
    of these will only run the one line of code.}

  \begin{itemize}
  \tightlist
  \item
    Click \texttt{Run} at the top left of the editor window
  \item
    Press \texttt{Ctrl\ +\ enter}
  \end{itemize}
\item
  Highlight the code that you would like to run and follow the options
  in step 1 above.
\item
  \texttt{Ctrl\ +\ Shift\ +\ s} sends all code from the editor to be
  evaluated.
\end{enumerate}

\marginnote{I tend to prefer using 'Ctrl + Enter' as I like using keyboard shortcuts. If you want to see all available keyboard shortcuts either go to Help and choose keyboard shortcuts, alternatively 'Alt + Shift + k' is the keyboard shortcut for the keyboard shortcut menu.}

\hypertarget{course-r-package}{%
\section{Course R package}\label{course-r-package}}

Make sure that you have the course package loaded into the current
session. Instructions on how to install the package are contained in the
appendix to the course notes. You should only need to install a package
once using \texttt{install.packages()} and then we need to use
\texttt{library()} in every session or script that we want to use the
functionality of that package.

\begin{Shaded}
\begin{Highlighting}[]
\KeywordTok{library}\NormalTok{(}\StringTok{"jrIntroduction"}\NormalTok{)}
\end{Highlighting}
\end{Shaded}

\hypertarget{vectors}{%
\section{Vectors}\label{vectors}}

Write the following code in the editor and run it

\begin{Shaded}
\begin{Highlighting}[]
\NormalTok{x1 =}\StringTok{ }\KeywordTok{GetNumericVector}\NormalTok{()}
\end{Highlighting}
\end{Shaded}

\noindent This code generates a large vector of random numbers such that
everyone has the same.\footnote{This function is part of the course
  package \textbf{jrIntroduction}, you can view it's help page if you
  like \texttt{r\ ?GetNumericVector()}}

\begin{enumerate}
\def\labelenumi{\arabic{enumi}.}
\item
  What is the length of \texttt{x1}?
\item
  What is the 55\textsuperscript{th} element of \texttt{x1}?
\item
  What is the final value in \texttt{x1}?
\item
  What is the 50\textsuperscript{th} smallest value of \texttt{x1}?
\item
  How many unique values are there in \texttt{x1}?
\item
  What is the total of all elements?
\end{enumerate}

\hypertarget{solutions}{%
\section{Solutions}\label{solutions}}

Solutions to the practical questions are contained within the package

\begin{Shaded}
\begin{Highlighting}[]
\KeywordTok{vignette}\NormalTok{(}\StringTok{"solutions1"}\NormalTok{, }\DataTypeTok{package =} \StringTok{"jrIntroduction"}\NormalTok{)}
\end{Highlighting}
\end{Shaded}



\end{document}
