\documentclass[]{tufte-handout}

% ams
\usepackage{amssymb,amsmath}

\usepackage{ifxetex,ifluatex}
\usepackage{fixltx2e} % provides \textsubscript
\ifnum 0\ifxetex 1\fi\ifluatex 1\fi=0 % if pdftex
  \usepackage[T1]{fontenc}
  \usepackage[utf8]{inputenc}
\else % if luatex or xelatex
  \makeatletter
  \@ifpackageloaded{fontspec}{}{\usepackage{fontspec}}
  \makeatother
  \defaultfontfeatures{Ligatures=TeX,Scale=MatchLowercase}
  \makeatletter
  \@ifpackageloaded{soul}{
     \renewcommand\allcapsspacing[1]{{\addfontfeature{LetterSpace=15}#1}}
     \renewcommand\smallcapsspacing[1]{{\addfontfeature{LetterSpace=10}#1}}
   }{}
  \makeatother

\fi

% graphix
\usepackage{graphicx}
\setkeys{Gin}{width=\linewidth,totalheight=\textheight,keepaspectratio}

% booktabs
\usepackage{booktabs}

% url
\usepackage{url}

% hyperref
\usepackage{hyperref}

% units.
\usepackage{units}


\setcounter{secnumdepth}{-1}

% citations

% pandoc syntax highlighting
\usepackage{color}
\usepackage{fancyvrb}
\newcommand{\VerbBar}{|}
\newcommand{\VERB}{\Verb[commandchars=\\\{\}]}
\DefineVerbatimEnvironment{Highlighting}{Verbatim}{commandchars=\\\{\}}
% Add ',fontsize=\small' for more characters per line
\newenvironment{Shaded}{}{}
\newcommand{\AlertTok}[1]{\textcolor[rgb]{1.00,0.00,0.00}{\textbf{#1}}}
\newcommand{\AnnotationTok}[1]{\textcolor[rgb]{0.38,0.63,0.69}{\textbf{\textit{#1}}}}
\newcommand{\AttributeTok}[1]{\textcolor[rgb]{0.49,0.56,0.16}{#1}}
\newcommand{\BaseNTok}[1]{\textcolor[rgb]{0.25,0.63,0.44}{#1}}
\newcommand{\BuiltInTok}[1]{#1}
\newcommand{\CharTok}[1]{\textcolor[rgb]{0.25,0.44,0.63}{#1}}
\newcommand{\CommentTok}[1]{\textcolor[rgb]{0.38,0.63,0.69}{\textit{#1}}}
\newcommand{\CommentVarTok}[1]{\textcolor[rgb]{0.38,0.63,0.69}{\textbf{\textit{#1}}}}
\newcommand{\ConstantTok}[1]{\textcolor[rgb]{0.53,0.00,0.00}{#1}}
\newcommand{\ControlFlowTok}[1]{\textcolor[rgb]{0.00,0.44,0.13}{\textbf{#1}}}
\newcommand{\DataTypeTok}[1]{\textcolor[rgb]{0.56,0.13,0.00}{#1}}
\newcommand{\DecValTok}[1]{\textcolor[rgb]{0.25,0.63,0.44}{#1}}
\newcommand{\DocumentationTok}[1]{\textcolor[rgb]{0.73,0.13,0.13}{\textit{#1}}}
\newcommand{\ErrorTok}[1]{\textcolor[rgb]{1.00,0.00,0.00}{\textbf{#1}}}
\newcommand{\ExtensionTok}[1]{#1}
\newcommand{\FloatTok}[1]{\textcolor[rgb]{0.25,0.63,0.44}{#1}}
\newcommand{\FunctionTok}[1]{\textcolor[rgb]{0.02,0.16,0.49}{#1}}
\newcommand{\ImportTok}[1]{#1}
\newcommand{\InformationTok}[1]{\textcolor[rgb]{0.38,0.63,0.69}{\textbf{\textit{#1}}}}
\newcommand{\KeywordTok}[1]{\textcolor[rgb]{0.00,0.44,0.13}{\textbf{#1}}}
\newcommand{\NormalTok}[1]{#1}
\newcommand{\OperatorTok}[1]{\textcolor[rgb]{0.40,0.40,0.40}{#1}}
\newcommand{\OtherTok}[1]{\textcolor[rgb]{0.00,0.44,0.13}{#1}}
\newcommand{\PreprocessorTok}[1]{\textcolor[rgb]{0.74,0.48,0.00}{#1}}
\newcommand{\RegionMarkerTok}[1]{#1}
\newcommand{\SpecialCharTok}[1]{\textcolor[rgb]{0.25,0.44,0.63}{#1}}
\newcommand{\SpecialStringTok}[1]{\textcolor[rgb]{0.73,0.40,0.53}{#1}}
\newcommand{\StringTok}[1]{\textcolor[rgb]{0.25,0.44,0.63}{#1}}
\newcommand{\VariableTok}[1]{\textcolor[rgb]{0.10,0.09,0.49}{#1}}
\newcommand{\VerbatimStringTok}[1]{\textcolor[rgb]{0.25,0.44,0.63}{#1}}
\newcommand{\WarningTok}[1]{\textcolor[rgb]{0.38,0.63,0.69}{\textbf{\textit{#1}}}}

% longtable

% multiplecol
\usepackage{multicol}

% strikeout
\usepackage[normalem]{ulem}

% morefloats
\usepackage{morefloats}


% tightlist macro required by pandoc >= 1.14
\providecommand{\tightlist}{%
  \setlength{\itemsep}{0pt}\setlength{\parskip}{0pt}}

% title / author / date
\title{Practical 3}
\author{Jumping Rivers}
\date{}


\begin{document}

\maketitle




\hypertarget{the-ggplot2-package}{%
\section{The ggplot2 Package}\label{the-ggplot2-package}}

We will continue to investigate the movies data from earlier. Make sure
that you have the data loaded into the session as part of your new
script (if you started one) and the package loaded.

\begin{Shaded}
\begin{Highlighting}[]
\KeywordTok{library}\NormalTok{(}\StringTok{"ggplot2"}\NormalTok{)}
\KeywordTok{data}\NormalTok{(movies, }\DataTypeTok{package =} \StringTok{"jrIntroduction"}\NormalTok{)}
\end{Highlighting}
\end{Shaded}

\hypertarget{scatter-plots}{%
\subsection{Scatter plots}\label{scatter-plots}}

\begin{enumerate}
\def\labelenumi{\arabic{enumi}.}
\item
  Make a basic scatter plot with votes on the x-axis and rating on the
  y-axis. Remember, we use the \texttt{ggplot()} function to create a
  basic plot and them \texttt{geom\_point()} to add points.
\item
  Use the \texttt{labs()} function to change the axis labels \& and
  title to something better
\item
  The range of possible ratings is between 0 and 10, however because the
  maximum rating is below 10 the y-axis stops before 10. We can change
  the axis range using the \texttt{ylim()} function. For instance, if
  our graph was stored in an object \texttt{g}, we could change the
  y-axis limit to (0, 10) by adding \texttt{ylim(0,\ 10)}. Make this
  change to your graph.
\end{enumerate}

\hypertarget{histograms}{%
\subsection{Histograms}\label{histograms}}

\begin{enumerate}
\def\labelenumi{\arabic{enumi}.}
\item
  Make a basic histogram of the year of releases.
\item
  Change the binwidth to 1 year using the \texttt{binwidth} argument in
  \texttt{geom\_histogram(binwidth\ =\ 1)}. Try different values of
  \texttt{binwidth}.
\end{enumerate}

\hypertarget{bonus-questions---an-introduction-to-colours-and-fills}{%
\subsection{Bonus questions - an introduction to colours and
fills}\label{bonus-questions---an-introduction-to-colours-and-fills}}

Let's go back to the classification bar chart we had in the notes

\begin{enumerate}
\def\labelenumi{\arabic{enumi}.}
\item
  Inside the \texttt{geom\_bar()} function, try adding the argument
  \texttt{colour\ =\ "blue"}. What happens?
\item
  Change \texttt{colour\ =\ "blue"} to \texttt{fill\ =\ "blue"}?
\item
  Experiment with other colours, you can find a list of colours that R
  takes using the \texttt{colours()} function i.e.~run
\item
  Remove \texttt{fill\ =\ "blue"} from \texttt{geom\_bar()}. Now try
  adding \texttt{fill\ =\ classification} to the \texttt{aes()}
  function. What happens?
\end{enumerate}

Again, \textbf{ggplot2} is a huge package and so we can only cover the
basics today. This is just a general intro into some of the main
concepts.

\hypertarget{solutions}{%
\section{Solutions}\label{solutions}}

Solutions to the practical questions are contained within the package

\begin{Shaded}
\begin{Highlighting}[]
\KeywordTok{vignette}\NormalTok{(}\StringTok{"solutions3"}\NormalTok{, }\DataTypeTok{package =} \StringTok{"jrIntroduction"}\NormalTok{)}
\end{Highlighting}
\end{Shaded}



\end{document}
