\documentclass[]{tufte-handout}

% ams
\usepackage{amssymb,amsmath}

\usepackage{ifxetex,ifluatex}
\usepackage{fixltx2e} % provides \textsubscript
\ifnum 0\ifxetex 1\fi\ifluatex 1\fi=0 % if pdftex
  \usepackage[T1]{fontenc}
  \usepackage[utf8]{inputenc}
\else % if luatex or xelatex
  \makeatletter
  \@ifpackageloaded{fontspec}{}{\usepackage{fontspec}}
  \makeatother
  \defaultfontfeatures{Ligatures=TeX,Scale=MatchLowercase}
  \makeatletter
  \@ifpackageloaded{soul}{
     \renewcommand\allcapsspacing[1]{{\addfontfeature{LetterSpace=15}#1}}
     \renewcommand\smallcapsspacing[1]{{\addfontfeature{LetterSpace=10}#1}}
   }{}
  \makeatother

\fi

% graphix
\usepackage{graphicx}
\setkeys{Gin}{width=\linewidth,totalheight=\textheight,keepaspectratio}

% booktabs
\usepackage{booktabs}

% url
\usepackage{url}

% hyperref
\usepackage{hyperref}

% units.
\usepackage{units}


\setcounter{secnumdepth}{-1}

% citations

% pandoc syntax highlighting
\usepackage{color}
\usepackage{fancyvrb}
\newcommand{\VerbBar}{|}
\newcommand{\VERB}{\Verb[commandchars=\\\{\}]}
\DefineVerbatimEnvironment{Highlighting}{Verbatim}{commandchars=\\\{\}}
% Add ',fontsize=\small' for more characters per line
\newenvironment{Shaded}{}{}
\newcommand{\AlertTok}[1]{\textcolor[rgb]{1.00,0.00,0.00}{\textbf{#1}}}
\newcommand{\AnnotationTok}[1]{\textcolor[rgb]{0.38,0.63,0.69}{\textbf{\textit{#1}}}}
\newcommand{\AttributeTok}[1]{\textcolor[rgb]{0.49,0.56,0.16}{#1}}
\newcommand{\BaseNTok}[1]{\textcolor[rgb]{0.25,0.63,0.44}{#1}}
\newcommand{\BuiltInTok}[1]{#1}
\newcommand{\CharTok}[1]{\textcolor[rgb]{0.25,0.44,0.63}{#1}}
\newcommand{\CommentTok}[1]{\textcolor[rgb]{0.38,0.63,0.69}{\textit{#1}}}
\newcommand{\CommentVarTok}[1]{\textcolor[rgb]{0.38,0.63,0.69}{\textbf{\textit{#1}}}}
\newcommand{\ConstantTok}[1]{\textcolor[rgb]{0.53,0.00,0.00}{#1}}
\newcommand{\ControlFlowTok}[1]{\textcolor[rgb]{0.00,0.44,0.13}{\textbf{#1}}}
\newcommand{\DataTypeTok}[1]{\textcolor[rgb]{0.56,0.13,0.00}{#1}}
\newcommand{\DecValTok}[1]{\textcolor[rgb]{0.25,0.63,0.44}{#1}}
\newcommand{\DocumentationTok}[1]{\textcolor[rgb]{0.73,0.13,0.13}{\textit{#1}}}
\newcommand{\ErrorTok}[1]{\textcolor[rgb]{1.00,0.00,0.00}{\textbf{#1}}}
\newcommand{\ExtensionTok}[1]{#1}
\newcommand{\FloatTok}[1]{\textcolor[rgb]{0.25,0.63,0.44}{#1}}
\newcommand{\FunctionTok}[1]{\textcolor[rgb]{0.02,0.16,0.49}{#1}}
\newcommand{\ImportTok}[1]{#1}
\newcommand{\InformationTok}[1]{\textcolor[rgb]{0.38,0.63,0.69}{\textbf{\textit{#1}}}}
\newcommand{\KeywordTok}[1]{\textcolor[rgb]{0.00,0.44,0.13}{\textbf{#1}}}
\newcommand{\NormalTok}[1]{#1}
\newcommand{\OperatorTok}[1]{\textcolor[rgb]{0.40,0.40,0.40}{#1}}
\newcommand{\OtherTok}[1]{\textcolor[rgb]{0.00,0.44,0.13}{#1}}
\newcommand{\PreprocessorTok}[1]{\textcolor[rgb]{0.74,0.48,0.00}{#1}}
\newcommand{\RegionMarkerTok}[1]{#1}
\newcommand{\SpecialCharTok}[1]{\textcolor[rgb]{0.25,0.44,0.63}{#1}}
\newcommand{\SpecialStringTok}[1]{\textcolor[rgb]{0.73,0.40,0.53}{#1}}
\newcommand{\StringTok}[1]{\textcolor[rgb]{0.25,0.44,0.63}{#1}}
\newcommand{\VariableTok}[1]{\textcolor[rgb]{0.10,0.09,0.49}{#1}}
\newcommand{\VerbatimStringTok}[1]{\textcolor[rgb]{0.25,0.44,0.63}{#1}}
\newcommand{\WarningTok}[1]{\textcolor[rgb]{0.38,0.63,0.69}{\textbf{\textit{#1}}}}

% longtable

% multiplecol
\usepackage{multicol}

% strikeout
\usepackage[normalem]{ulem}

% morefloats
\usepackage{morefloats}


% tightlist macro required by pandoc >= 1.14
\providecommand{\tightlist}{%
  \setlength{\itemsep}{0pt}\setlength{\parskip}{0pt}}

% title / author / date
\title{Practical 2}
\author{Jumping Rivers}
\date{}


\begin{document}

\maketitle




\hypertarget{a-suggestion}{%
\subsection{A suggestion}\label{a-suggestion}}

This is just a suggestion to try and help get you into ``good habits''
early. If I was sitting to take this practical now, I would start a new
R script file. That way all of the work that you have done associated
with today's course is in one place, and the code for each of the
practicals is separate from one another. This might feel a bit tedious
right now, but as the amount of code you write and number of projects
you take part in increases it will pay off to have a structured
workflow.

\hypertarget{tibbles}{%
\section{Tibbles}\label{tibbles}}

For this set of questions we will use the movies data from the IMDB
database.

\begin{enumerate}
\def\labelenumi{\arabic{enumi}.}
\item
  Use the \texttt{head()} function to inspect the top of the data, this
  can help give you a feel for what the data looks like and what
  variables are contained within the data
\item
  How many films and how many variables are in this data set?
\item
  Recall that if I want to pull out a single column, or variable, from a
  data frame we can use \texttt{\$}. To extract the titles from this
  data set we write \texttt{movies\$title}.\footnote{If you can't
    remember what the names of the columns are, you can use
    \texttt{colnames(movies)} to find out.} Using \texttt{mean()} and
  \texttt{median()} calculate these summary statistics for the film
  lengths
\item
  What year is the oldest film in the data set from?
\item
  How long is the longest film?
\item
  What is the standard deviation of the film ratings?
\item
  Try running the following code

\begin{Shaded}
\begin{Highlighting}[]
\KeywordTok{table}\NormalTok{(movies}\OperatorTok{$}\NormalTok{action)}
\end{Highlighting}
\end{Shaded}

  What do you think is happening?
\end{enumerate}

\hypertarget{loading-data-from-a-csv-file}{%
\subsection{Loading data from a CSV
file}\label{loading-data-from-a-csv-file}}

To give you some practice at reading in your own data, we're going to
get you to read it in. The function

\begin{Shaded}
\begin{Highlighting}[]
\KeywordTok{library}\NormalTok{(}\StringTok{"jrIntroduction"}\NormalTok{)}
\KeywordTok{get_csv_movies_file}\NormalTok{()}
\end{Highlighting}
\end{Shaded}

\noindent will copy a file called \texttt{movies.csv} into your current
working directory. You can now import the data set using the
\emph{Import Dataset} button in RStudio, under the Environment tab

\includegraphics[width=0.6\textwidth]{/home/john/R/x86_64-pc-linux-gnu-library/3.6/jrIntroduction/import_data}

\noindent This will generate R code that you can reuse.

\hypertarget{solutions}{%
\section{Solutions}\label{solutions}}

Solutions to the practical questions are contained within the package

\begin{Shaded}
\begin{Highlighting}[]
\KeywordTok{vignette}\NormalTok{(}\StringTok{"solutions2"}\NormalTok{, }\DataTypeTok{package =} \StringTok{"jrIntroduction"}\NormalTok{)}
\end{Highlighting}
\end{Shaded}



\end{document}
