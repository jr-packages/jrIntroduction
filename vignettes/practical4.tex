\documentclass[]{tufte-handout}

% ams
\usepackage{amssymb,amsmath}

\usepackage{ifxetex,ifluatex}
\usepackage{fixltx2e} % provides \textsubscript
\ifnum 0\ifxetex 1\fi\ifluatex 1\fi=0 % if pdftex
  \usepackage[T1]{fontenc}
  \usepackage[utf8]{inputenc}
\else % if luatex or xelatex
  \makeatletter
  \@ifpackageloaded{fontspec}{}{\usepackage{fontspec}}
  \makeatother
  \defaultfontfeatures{Ligatures=TeX,Scale=MatchLowercase}
  \makeatletter
  \@ifpackageloaded{soul}{
     \renewcommand\allcapsspacing[1]{{\addfontfeature{LetterSpace=15}#1}}
     \renewcommand\smallcapsspacing[1]{{\addfontfeature{LetterSpace=10}#1}}
   }{}
  \makeatother

\fi

% graphix
\usepackage{graphicx}
\setkeys{Gin}{width=\linewidth,totalheight=\textheight,keepaspectratio}

% booktabs
\usepackage{booktabs}

% url
\usepackage{url}

% hyperref
\usepackage{hyperref}

% units.
\usepackage{units}


\setcounter{secnumdepth}{-1}

% citations

% pandoc syntax highlighting
\usepackage{color}
\usepackage{fancyvrb}
\newcommand{\VerbBar}{|}
\newcommand{\VERB}{\Verb[commandchars=\\\{\}]}
\DefineVerbatimEnvironment{Highlighting}{Verbatim}{commandchars=\\\{\}}
% Add ',fontsize=\small' for more characters per line
\newenvironment{Shaded}{}{}
\newcommand{\AlertTok}[1]{\textcolor[rgb]{1.00,0.00,0.00}{\textbf{#1}}}
\newcommand{\AnnotationTok}[1]{\textcolor[rgb]{0.38,0.63,0.69}{\textbf{\textit{#1}}}}
\newcommand{\AttributeTok}[1]{\textcolor[rgb]{0.49,0.56,0.16}{#1}}
\newcommand{\BaseNTok}[1]{\textcolor[rgb]{0.25,0.63,0.44}{#1}}
\newcommand{\BuiltInTok}[1]{#1}
\newcommand{\CharTok}[1]{\textcolor[rgb]{0.25,0.44,0.63}{#1}}
\newcommand{\CommentTok}[1]{\textcolor[rgb]{0.38,0.63,0.69}{\textit{#1}}}
\newcommand{\CommentVarTok}[1]{\textcolor[rgb]{0.38,0.63,0.69}{\textbf{\textit{#1}}}}
\newcommand{\ConstantTok}[1]{\textcolor[rgb]{0.53,0.00,0.00}{#1}}
\newcommand{\ControlFlowTok}[1]{\textcolor[rgb]{0.00,0.44,0.13}{\textbf{#1}}}
\newcommand{\DataTypeTok}[1]{\textcolor[rgb]{0.56,0.13,0.00}{#1}}
\newcommand{\DecValTok}[1]{\textcolor[rgb]{0.25,0.63,0.44}{#1}}
\newcommand{\DocumentationTok}[1]{\textcolor[rgb]{0.73,0.13,0.13}{\textit{#1}}}
\newcommand{\ErrorTok}[1]{\textcolor[rgb]{1.00,0.00,0.00}{\textbf{#1}}}
\newcommand{\ExtensionTok}[1]{#1}
\newcommand{\FloatTok}[1]{\textcolor[rgb]{0.25,0.63,0.44}{#1}}
\newcommand{\FunctionTok}[1]{\textcolor[rgb]{0.02,0.16,0.49}{#1}}
\newcommand{\ImportTok}[1]{#1}
\newcommand{\InformationTok}[1]{\textcolor[rgb]{0.38,0.63,0.69}{\textbf{\textit{#1}}}}
\newcommand{\KeywordTok}[1]{\textcolor[rgb]{0.00,0.44,0.13}{\textbf{#1}}}
\newcommand{\NormalTok}[1]{#1}
\newcommand{\OperatorTok}[1]{\textcolor[rgb]{0.40,0.40,0.40}{#1}}
\newcommand{\OtherTok}[1]{\textcolor[rgb]{0.00,0.44,0.13}{#1}}
\newcommand{\PreprocessorTok}[1]{\textcolor[rgb]{0.74,0.48,0.00}{#1}}
\newcommand{\RegionMarkerTok}[1]{#1}
\newcommand{\SpecialCharTok}[1]{\textcolor[rgb]{0.25,0.44,0.63}{#1}}
\newcommand{\SpecialStringTok}[1]{\textcolor[rgb]{0.73,0.40,0.53}{#1}}
\newcommand{\StringTok}[1]{\textcolor[rgb]{0.25,0.44,0.63}{#1}}
\newcommand{\VariableTok}[1]{\textcolor[rgb]{0.10,0.09,0.49}{#1}}
\newcommand{\VerbatimStringTok}[1]{\textcolor[rgb]{0.25,0.44,0.63}{#1}}
\newcommand{\WarningTok}[1]{\textcolor[rgb]{0.38,0.63,0.69}{\textbf{\textit{#1}}}}

% longtable

% multiplecol
\usepackage{multicol}

% strikeout
\usepackage[normalem]{ulem}

% morefloats
\usepackage{morefloats}


% tightlist macro required by pandoc >= 1.14
\providecommand{\tightlist}{%
  \setlength{\itemsep}{0pt}\setlength{\parskip}{0pt}}

% title / author / date
\title{Practical 4}
\author{Jumping Rivers}
\date{}


\begin{document}

\maketitle




\hypertarget{data-manipulation}{%
\section{Data Manipulation}\label{data-manipulation}}

This section will hopefully help you get more comfortable with some of
the \textbf{dplyr} functionality for ``wrangling'' your data. We will do
some data wrangling and the use that to create some plots. Make sure you
load in the \textbf{dplyr} package and the \texttt{movies} data set.

\begin{Shaded}
\begin{Highlighting}[]
\KeywordTok{library}\NormalTok{(}\StringTok{"dplyr"}\NormalTok{)}
\KeywordTok{data}\NormalTok{(}\StringTok{"movies"}\NormalTok{, }\DataTypeTok{package =} \StringTok{"jrIntroduction"}\NormalTok{)}
\end{Highlighting}
\end{Shaded}

\begin{enumerate}
\def\labelenumi{\arabic{enumi}.}
\item
  We want to look at how film budgets for films in English have changed
  over time for both the Comedy and non Comedy films. To start with, we
  should filter the data set such that it only contains films spoken in
  English. Try using the \texttt{\%\textgreater{}\%} notation, i.e.
  \texttt{movies\ \%\textgreater{}\%\ filter(...)}
\item
  We want to look at comedy and non comedy films in each year, this is
  some sort of grouping structure which suggests use of the
  \texttt{group\_by()} function. Create this grouping structure on the
  filtered movies data
\item
  Use the \texttt{summarise()} function to calculate the average budget
  in each year for both comedy and non comedy films.
\end{enumerate}

\hypertarget{question-2}{%
\subsection{Question 2}\label{question-2}}

Run the folowing R code:

\begin{Shaded}
\begin{Highlighting}[]
\KeywordTok{data}\NormalTok{(USnames, }\DataTypeTok{package =} \StringTok{"jrIntroduction"}\NormalTok{)}
\end{Highlighting}
\end{Shaded}

\noindent The tibble \texttt{USnames} is a collection of names given to
babies born in the US between 2011 and 2014.

\begin{enumerate}
\def\labelenumi{\arabic{enumi}.}
\item
  Make sure you are comfortable with what the data looks like using
  \texttt{head()} and \texttt{colnames()}.
\item
  How many children were born in 2012? Hint: use \texttt{filter()} then
  \texttt{summarise()}
\item
  Were more male or females children born during the four years? Hint:
  use \texttt{group\_by()}
\item
  Tricky: How many names in 2011 were used fewer than ten times?
\end{enumerate}

\hypertarget{solutions}{%
\section{Solutions}\label{solutions}}

Solutions to the practical questions are contained within the package

\begin{Shaded}
\begin{Highlighting}[]
\KeywordTok{vignette}\NormalTok{(}\StringTok{"solutions4"}\NormalTok{, }\DataTypeTok{package =} \StringTok{"jrIntroduction"}\NormalTok{)}
\end{Highlighting}
\end{Shaded}



\end{document}
